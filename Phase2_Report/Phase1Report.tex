\documentclass{article}

\usepackage{hyperref}
\usepackage[margin=1in]{geometry}
\usepackage{enumitem}
\usepackage{float}

\usepackage{graphicx}
\graphicspath{{./}}

\usepackage{fancyhdr}
\pagestyle{fancy}

\lhead{null'); DROP TABLE Students;}
\chead{\Large \textbf{Phase 1}\vspace{0pt}}
\rhead{\today}
\setlength{\headheight}{46pt}
\setlength{\headsep}{4ex}

\renewcommand{\arraystretch}{1.3}
\setlist[itemize, 2]{label={--}}


\title{Principles of Data Management\\\Large Phase 1}
\date{\today}
\author{
	Andrew Hamrick
	\and
	Daniyal Iqbal
	\and
	Oscar Lowry
	\and
	Qasim Ali
}

\begin{document}
	\maketitle
	\newpage

	\section{Design}
		Our database was designed with a Retail environment in mind. The choices for
		entities and relations were made by considering what retail employees and
		stores would likely need to check or keep track of. The retail environment
		envisioned focuses on electronic devices. Currently our design supports
		Computers and TVs but can easily be expanded to other product types. We
		decided to focus on computers and TVs as that is an area of expertise for
		us.\\
		\vspace{0pt}

		Our database centers around the product entity as that is largely seen as
		the most important component of a retail environment. The other entities
		were chosen to track the flow of money in and out of the stores, or to
		support the product. Support for the products includes keeping track of
		current inventory, acquiring more of the product, and categorizing the
		product into product types.

	\section{ER Diagram}
		The Product entity is a major entity in our database. It has a relation with
		almost every entity. This makes sense for a retail database as retailers are
		generally focused on the products they sell. The major products sold at our
		establishment are TVs, Laptops, and Desktop computers. Our product types are
		Screen and Computer as those make up the two major groups of our products.
		These subgroups inherit identical attributes from parent entities.  TV
		obviously inherits from Screen. As the screen is a major component of a
		laptop, laptop also inherits from Screen.  Desktop obviously inherits from
		Computer, however because a laptop is also a computer, it inherits from
		Computer in addition to Screen.

		\noindent 
		The Tables for our database are:
		\begin{itemize}
			\item Product(\underline{UPC}, Description, Brand);
			\item Laptop(\underline{UPC}, Battery\_Life);
			\item Brand(\underline{Name}, Trademark, Vendor);
			\item Order(\underline{ID}, Date\_placed, Date\_arrived, Cost, Tracking\_number, Vendor\_name, UPC, Quantity);
			\item Transaction(\underline{ID}, Date, Total, Payment\_Method, Customer\_ID, Store\_ID);
			\item Bought\_Products(Product\_UPC, Transaction\_ID, Quantity, Price);
			\item Store(\underline{ID}, Phone Number, Date Opened, Address);
			\item Stock(Store\_ID, UPC, Inventory, Listed\_Price);
			\item Desktop(\underline{UPC}, Power Supply, Graphics Card);
			\item Computer(\underline{UPC}, Ram Size, Processor, Hard Disk);
			\item TV(\underline{UPC}, Smart Features, Mount Type);
			\item Screen(\underline{UPC}, Size, Resolution, PPI, Panel Type);
			\item Customer(\underline{ID}, Name, Email Address, Date Joined, Phone\_Number\_1, Phone\_Number\_2);
			\item Vendor(\underline{Name}, Phone, Website);
		\end{itemize}

		\subsection{Products}
			Product contains the attributes UPC and Description. UPC is a unique
			identifier to identify the product in the database. Description is a short
			description of the product for customer and employee benefit. All other
			attributes associated with a product specified by that products subtype.
			Our subtypes are Compter and Monitor. All products must also have a UPC that 
			matches one in the product above it. For example, a Laptop must match a 
			computer that must match a product.

			\subsubsection{Monitor}
				Monitor was chosen as a product type due to our focus on electronics.  A
				large number of electronics these days feature a screen. In addition, the
				monitor is a major consideration in the purchase of an electronic device.

			\subsection{Computer}
				Computers are the other major section of the products represented by our
				database. Computers main common attributes among all types of computers
				are RAM size, the processor, and the Hard Drive size. 
				\begin{itemize}
				
					\item [Laptop] Laptop inherits from both Computer and Screen because 
					it fits well in both categories. All important parts of a Laptop are 
					covered in either Screen or Computer, except for Battery Life.
					
					\item [Desktop]  A desktop computer is just the tower. It does not 
					include a keyboard, a mouse, or a monitor, therefore it does not 
					inherit from Screen. Attributes that apply to desktops that don't apply 
					to other computers (Laptops) are the Power Supply, and the Graphics Card.
					
				\end{itemize}

		\subsection{Brand}
			Brand is an alternate way to group products together. The major aspects of
			Brand is the Name and the vendor it belongs to. The Brand names must not 
			match the brand name of any other brand, and the vendor must pair with 
			an existing vendor in the vendor table. 

		\subsection{Vendor}
			A vendor represents a company that our stores purchase products from to
			sell to consumers. A vendor is represented by their company name, their
			phone number, and their website. A vendor is identified by their name as
			that is how they are identified.

		\subsection{Shipment}
			Shipment represents an shipment placed by an employee to resupply a stores
			stock. The shipment consists of a unique ID to identify it, the date it was placed, 
			the date it arrived, the total cost of the shipment, the tracking number for the
			shipment, the vendor name, a UPC, and the quantity, and a store ID. The quantity 
			and UPC are used to keep track of how many of each product was shipment. 
			The vendor name is used to link the shipment with the vendor. The store id is 
			used to keep track of the store the product is being delivered to.

		\subsection{Bought\_Products}
			A table was needed to handle the relation between Purchase and Product
			as multiple of the same product could be purchased in one purchase.
			Price also needed to be tracked as prices of a particular product change
			at different stores.

		\subsection{Store}
			Our database may need to track information about multiple stores. A store
			is represented by a unique ID, the store's phone number, the date the
			store opened, a yearly budget for the store, and the store's address.

		\subsection{Stock}
			A table is needed to manage interactions between Stores and Products.
			A store will sell multiple products, and a product may be stored at
			multiple stores. Stores will have varying inventory of a certain product,
			and may also vary the products type. Stock is represented by a store's
			unique ID, the UPC of a product, how much of that product is currently in
			the store, and the price the store is selling the product for.

		\subsection{Customer}
			Our stores offer a customer loyalty program. This allows customers to be
			tied to the purchases they make. A customer is represented by a unique ID,
			their name, their email address, the date they joined the program, and
			their phone number(s).

		\subsection{Purchase}
			A purchase consists of a unique ID, the date it was placed, the payment
			total, the method of payment (e.g. Credit Card, Debit Card, Cash, etc.), a
			customer ID (e.g. phone number, etc.), and the store ID. A purchase can
			be tied to a customer for the loyalty program. A purchase will not be
			tied to a customer if the customer is not a part of the loyalty program.
			Due to the loyalty program it is likely a customer will be tied to
			multiple purchase, but a single purchase will only be tied to a
			single customer.



	\section{Test Data}
		Our test data was generated to fit the tables made from our ER diagram. The
		data was mostly generated through a random data generator
		(\url{https://www.mockaroo.com/}) and manual construction. Data that was
		independent of other data (such as stores and customers) was generated by
		the random data generator. The rest of the data was generated manually
		and/or through the random data generator to maintain consistency across
		different tables.

	\section{User Interface}
    Our user interface is a command line application using straight forward
    commands. The current commands are as follows.

    \begin{itemize}
      \item
        Add: customer, transaction, store, product, order, vendor, brand\\
        Examples:
        \begin{itemize}
          \item Add customer name=``John Smith" email=john@gmail.com phone=5855551234
          \item Add transaction customer=123 product ``Sony TV 52" 2
        \end{itemize}

      \item
        Update: store, stock, price, customer\\
        Examples:
        \begin{itemize}
          \item Update store 123 phone 1235557890
          \item Update product ``Dell Laptop 15" price=1000 store=123
        \end{itemize}

        \pagebreak

      \item 
        Lowest Inventory:
        \begin{itemize}
          \item Product
          \item Number of low inventory products to show
          \item Sort field: inventory number, expected days until out of stock
          \item Search field: stores
        \end{itemize}
        Examples:
        \begin{itemize}
          \item Low 50 products store=123
          \item Low 5 days store=all
        \end{itemize}

      \item
        Top/Bot: Number of top or bottom records
        \begin{itemize}
          \item Search term: products, product type, stores, brand, vendor
          \item Sort field: revenue, total items sold
          \item Criteria: product type, month/year, store
        \end{itemize}
        Examples:
        \begin{itemize}
          \item Top 5 stores items year=2017
          \item Bot 10 computers revenue store=12
          \item Top 1 brand items product=computer
          \item Top 5 products revenue
        \end{itemize}

      \item
        Compare:
        \begin{itemize}
          \item Comparison of: store, state
          \item Comparison term: brands, products, vendors
          \item Comparison field: revenue, items sold
        \end{itemize}
        Examples:
        \begin{itemize}
          \item Compare stores products laptops computers revenue 
          \item Compare states brands Dell HP items
        \end{itemize}

      \item
        Track: Show items commonly bought together
        \begin{itemize}
          \item Number of records to show
          \item Products, product type
        \end{itemize}
        Examples:
        \begin{itemize}
          \item Track 2 Desktop
          \item Track 10 ``LG TV 72"
        \end{itemize}
    \end{itemize}

  \end{document}
